%----------------------------------------------------------------------------
% Magic Maintainer's Manual #3
%----------------------------------------------------------------------------

\NeedsTeXFormat{LaTeX2e}[1994/12/01]
\documentclass[letterpaper,twoside,12pt]{article}
\usepackage{epsfig,times}

\setlength{\textwidth}{8.5in}
\addtolength{\textwidth}{-2.0in}
\setlength{\textheight}{11.0in}
\addtolength{\textheight}{-2.0in}
\setlength{\oddsidemargin}{0in}
\setlength{\evensidemargin}{0pt}
\setlength{\topmargin}{-0.5in}
\setlength{\headheight}{0.2in}
\setlength{\headsep}{0.3in}
\setlength{\topskip}{0pt}

\def\hinch{\hspace*{0.5in}}
\def\starti{\begin{center}\begin{tabbing}\hinch\=\hinch\=\hinch\=hinch\hinch\=\kill}
\def\endi{\end{tabbing}\end{center}}
\def\ii{\>\>\>}
\def\mytitle{Magic Maintainer's Manual \#3: Display Styles, Colormaps, and Glyphs}

%----------------------------------------------------------------------------

\begin{document}

\makeatletter
\newcommand{\ps@magic}{%
	\renewcommand{\@oddhead}{\mytitle\hfil\today}%
	\renewcommand{\@evenhead}{\today\hfil\mytitle}%
	\renewcommand{\@evenfoot}{\hfil\textrm{--{\thepage}--}\hfil}%
	\renewcommand{\@oddfoot}{\@evenfoot}}
\newcommand{\ps@mplain}{%
	\renewcommand{\@oddhead}{}%
	\renewcommand{\@evenhead}{}%
	\renewcommand{\@evenfoot}{\hfil\textrm{--{\thepage}--}\hfil}%
	\renewcommand{\@oddfoot}{\@evenfoot}}
\makeatother
\pagestyle{magic}
\thispagestyle{mplain}


\begin{center}
  {\bfseries \Large \mytitle} \\
  \vspace*{0.5in}
  {\itshape Robert N. Mayo} \\
  {\itshape John Ousterhout} \\
  \vspace*{0.5in}
   Computer Science Division \\
   Electrical Engineering and Computer Sciences \\
   University of California \\
   Berkeley, CA  94720 \\
  \vspace*{0.25in}
  This tutorial corresponds to Magic version 7. \\
\end{center}
\vspace*{0.5in}

{\noindent\bfseries\large Tutorials to read first:}
\starti
   \> All of them.
\endi

{\noindent\bfseries\large Commands introduced in this tutorial:}
\starti
   \> {\itshape (None)}
\endi

{\noindent\bfseries\large Macros introduced in this tutorial:}

\starti
   \> {\itshape (None)}
\endi

\vspace*{0.75in}
\section{Introduction}

This document gives overall information about
the files that tell Magic how to display information
on the screen.  There are three types of files that
contain display information:  display styles files,
color-map files, and glyph files.

\section{Display Styles}

Display styles files describe how to draw rectangular areas
and text.  A single file contains a large number of display
styles.  Each display style contains two kinds of
information:  a) how to modify pixels (which bits of
the pixel should be changed and what their new value(s) should
be); and b) which pixels to modify.  Part b) consists of
things like ``fill the entire area,'' or ``modify only those
pixels in the area that are given by a particular stipple pattern,''
or ``draw a dashed-line around the area's outline.''
In the case of text, ``which pixels to modify'' is determined
by the font for the text, which is not part of the display
style, so the display style information for this is ignored.
See the manual page {\bfseries dstyle~(5)} for details on the format
of display styles files.

Display styles are designed to take into account both the characteristics
of certain technologies and the characteristics of certain displays.
For example, a bipolar process may require information to be displayed
very differently than a MOS process, and a black-and-white display
will be used much differently than a color display.  Thus there can
be many different display styles files, each corresponding to a particular
class of technologies and a class of displays.  The names of styles
files reflect these classes:  each display styles file has a name of
the form {\itshape x}{\bfseries .}{\itshape y}{\bfseries .dstyle},
where {\itshape x} is the technology
class (given by the {\bfseries styletype} line in the {\bfseries styles}
section of the technology file), and {\itshape y} is the class of display.
Each display driver knows its display class;  the driver initialization
routine sets an internal Magic variable with the display class to
use.  Right now we have two display styles files:  {\bfseries mos.7bit.dstyle}
and {\bfseries mos.bw.dstyle}.  Both files contain enough different
styles to handle a variety of MOS processes, including both nMOS
and CMOS (hence the {\bfseries mos} field).  {\bfseries Mos.7bit.dstyle} is
designed for color displays with at least seven bits of color per
pixel, while {\bfseries mos.bw.dstyle} is for black-and-white displays
(stipple patterns are used instead of colors).

\section{Color Maps}

The display styles file tells how to modify pixels, but this doesn't
completely specify the color that will be displayed on the screen
(unless the screen is black-and-white).  For color displays, the
pixel values are used to index into a {\itshape color map}, which contains
the red, green, and blue intensity values to use for each pixel
value.  The values for color maps are stored in color-map files
and can be edited using the color-map-editing window in Magic.
See {\bfseries cmap~(5)} for details on the format of color-map files.

Each display styles file uses a separate color map.  Unfortunately,
some monitors have slightly different phosphors than others;  this
will result in different colors if the same intensity values are used
for them.  To compensate for monitor differences, Magic
supports multiple color maps for each display style, depending on
the monitor being used.  The monitor type can be specified with
the {\bfseries -m} command line switch to Magic, with {\bfseries std}
as the default.  Color-map files have names
of the form {\itshape x}{\bfseries .}{\itshape y}{\bfseries .}{\itshape z}
{\bfseries .cmap}, where {\itshape x} and
{\itshape y} have the same meaning as for display styles and {\itshape z} is
the monitor type.  Over the last few years monitor phosphors appear
to have standardized quite a bit, so almost all monitors now work
well with the {\bfseries std} monitor type.  The color map {\bfseries mos.7bit.std.cmap}
is the standard one used at Berkeley.

\section{Transparent and Opaque Layers}

One of the key decisions in defining a set of display styles
for a color display is how to use the bits of a pixel (this section
doesn't apply to black-and-white displays).  One option is to use
a separate bit of each pixel (called a {\itshape bit plane}) for each mask
layer.  The advantage of this is that each possible combination of
layer overlaps results in a different pixel value, and hence
a different color (if you wish).  Thus, for example, if metal
and poly are represented with different bit planes, poly-without-metal,
metal-without-poly, poly-and-metal, and neither-poly-nor-metal will
each cause a different value to be stored in the pixel.  A different
color can be used to display each of these combinations.  Typically,
the colors are chosen to present an illusion of transparency:  the
poly-and-metal color is chosen to make it appear as if metal were
a transparent colored foil placed on top of poly.  You can see this
effect if you paint polysilicon, metal1, and metal2 on top of each
other in our standard technologies.

The problem with transparent layers is that they require many bits
per pixel.  Most color displays don't have enough planes to use
a different one for each mask layer.  Another option is to use
a group of planes together.  For example, three bits of a pixel
can be used to store seven mask layers plus background, with
each mask layer corresponding to one of the combinations of
the three bits.  The problem with this scheme is that there
is no way to represent overlaps:  where there is an overlap,
one of the layers must be displayed at the expense of the
others.  We call this scheme an {\itshape opaque} one since when it
is used it appears as if each layer is an opaque foil, with
the foils lying on top of each other in some priority order.
This makes it harder to see what's going on when there are
several mask layers in an area.

The display styles files we've designed for Magic use a combination
of these techniques to get as much transparency as possible.
For example, our {\bfseries mos.7bit.dstyle} file uses three bits
of the pixel in an opaque scheme to represent polysilicon,
diffusion, and various combinations of them such as transistors.
Two additional bits are used, one each, for the two metal layers,
so they are transparent with respect to each other and the
poly-diff combinations.  Thus, although only one poly-diff combination
can appear at each point, it's possible to see the overlaps between
each of these combinations and each combination of metal1 and metal2.
Furthermore, all of these styles are overridden if the
sixth bit of the pixel is set.  In this case the low
order five bits no longer correspond to mask layers;  they are
used for opaque layers for things like labels and cell bounding
boxes, and override any mask information.  Thus, for example,
when metal1 is displayed it only affects one bit plane, but
when labels are displayed, the entire low-order six bits of the
pixel are modified.  It's important that the opaque layers like
labels are drawn after the transparent things that they blot out;
this is guaranteed by giving them higher style numbers in
the display styles files.

Finally, the seventh bit of the pixel is used for
highlights like the box and the selection.  All 64 entries
in the color map corresponding to pixel values with this
bit set contain the same value, namely pure white.  This makes
the highlights appear opaque with respect to everything
else.  However, since they have their own bit plane which
is completely independent of anything else, they can be
drawn and erased without having to redraw any of the mask
information underneath.  This is why the box can be moved
relatively quickly.  On the other hand, if Magic erases
a label it must redraw all the mask information in the area
because the label shared pixel bits with the mask information.

Thus, the scheme we've been using for Magic is a hierarchical
combination of transparent and opaque layers.  This scheme is
defined almost entirely by the styles file, so you can try
other schemes if you wish.  However, you're likely to have problems
if you try anything too radically different;  we haven't tried
any schemes but the one currently being used so there are probably
some code dependencies on it.

For more information on transparent and opaque layers, see
the paper ``The User Interface and Implementation of an IC Layout
Editor,'' which appeared in {\itshape IEEE Transactions on CAD} in
July 1984.


\section{Glyphs}

Glyphs are small rectangular bit patterns that are used in two
places in Magic.  The primary use for glyphs is for programmable
cursors, such as the shapes that show you which corner of the
box you're moving and the various tools described in Tutorial \#3.
Each programmable cursor is stored as a glyph describing the
pattern to be displayed in the cursor.  The second use of glyphs
is by the window package:  the little arrow icons appearing at the
ends of scroll bars are stored as glyphs, as is the zoom box in
the lower-left corner of the window.  We may eventually use glyphs
in a menu interface (but don't hold your breath).

Glyphs are stored in ASCII glyph files, each of which can hold
one or more glyph patterns.  Each glyph is represented as a
pattern of characters representing the pixels in the glyph.
Each character selects a display style from the current
display styles file;  the display style indicates the color to use for
that pixel.  See the manual page {\bfseries glyphs~(5)}
for details on the syntax of glyphs files.  

The window glyphs are stored in files of the form
{\bfseries windows}{\itshape XX}{\bfseries .glyphs}.
The {\itshape XX} indicates how wide the glyphs are, and is set by the graphics
driver for a particular display.  We started out with a {\bfseries windows7.glyphs}
and a {\bfseries windows11.glyphs}.   Since then, display resolution has 
increased greatly so we have also created a {\bfseries windows14.glyphs} and a
{\bfseries windows22.glyphs}.  The positions of the various glyphs in
these files is important, and is defined in the {\bfseries window} module of
Magic.

Programmable cursors are stored in files named
{\itshape x}{\bfseries .glyphs}, where {\itshape x} is determined by
the device driver
for the display.  Displays capable of supporting full-color cursors
use {\bfseries color.glyphs};  displays that can only support monochrome
cursors used {\bfseries bw.glyphs}.  The order of the various glyphs
in these files is important.  It is defined by the files {\bfseries styles.h}
in the {\bfseries misc} module of Magic.

\end{document}
